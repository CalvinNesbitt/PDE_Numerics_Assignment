\documentclass[11pt]{article}
\usepackage[margin=1in]{geometry}          
\usepackage{graphicx}
\usepackage{amsthm, amsmath, amssymb}
\usepackage[utf8]{inputenc}
\usepackage[english]{babel}
\usepackage{enumitem}
\usepackage{setspace}\onehalfspacing
\usepackage[loose,nice]{units} %replace "nice" by "ugly" for units in upright fractions
 
\title{PDE Numerics Asssignment}
\author{Calvin Nesbitt}
\date{}

\newtheorem{theorem}{Theorem}[section]
\newtheorem{corollary}{Corollary}[theorem]
\newtheorem{lemma}[theorem]{Lemma}
\newtheorem{definition}{Definition}

\newcommand{\Pro}{\mathbf{P}} % Shortcut for the Probability symbol
\newcommand{\A}{\mathcal{A}} % Shortcut for Algebra
\newcommand{\comp}[1]{{#1}^{\mathsf{c}}}
 
\begin{document}
\maketitle

\section{Introduction}
I have coded numerical schemes of the Linear Advection Equation. First I will briefly describe the linear advection equation.
\subsection{Schemes I will analyse}
Here I will briefly describe each scheme I have coded: FTBS, FTCS, CTCS, BTCS, Semi Lagrangian.
\subsection{Description of Analysis and Expectations}
Here I will briefly note that I will analyse Stability and Mass Conservation. I will outline my expectation that the Semi Lagrangian scheme performs best. I also expect BTCS and CTCS to perform well whilst I expect FTBS and FTCS to perform poorly.
\section{Stability Analysis}

Brief outline of stability and the L1, L2, LInf errors. Why we would expect them to change if scheme is unstable (Lax-Wendroff Thm). 

\subsection{Varying the Courant Number c}
Here I will describe the test I designed where I varied the courant number c and saw how the L1, L2 and Linf errors varied for each scheme. 
\subsection{Results}
Give results of test for each scheme. I will include plots of the errors for each scheme as functions of c. 
\subsection{Comparison to Theory}
Comparison of test results with theoretical expectation. Comment as to whether this is correct or not. Give reason if not correct.
\section{Mass Conservation Analysis}

Brief mention of mass conservation. Choose c where each scheme is still stable. 

\subsection{Mass Conservation Test}

Explain how I calculate the mass conservation of each scheme (take the mean after a fixed number of time steps.

\subsection{Results and Comparison to Theory}
Give results of test. Comparison of test results with theoretical expectation. Comment as to whether this is correct or not. Give reason if not correct.

\section{Scheme Comparison}

Conclusion about which schemes performed well which performed badly.

\section{Plots}
 
\end{document}